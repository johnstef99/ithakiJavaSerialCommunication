\documentclass{article}
\usepackage[LGR, T1]{fontenc}
\usepackage[utf8]{inputenc}
\usepackage[greek,english]{babel}
\usepackage{alphabeta}
\usepackage{hyperref}
\usepackage{tikz} % Draw
\usepackage{wrapfig} % Potision plots
\usepackage{pgfplots} % Plot
\pgfplotsset{compat = newest}
\usepackage{amsmath} % AMS Math Package
\usepackage{graphicx} % Allows for eps images

\usepackage{float}
\usepackage{subcaption}

\title{Δίκτυα Υπολογιστών Ι - Report}
\author{Στεφανίδης Ιωάννης}
\date{ΑΕΜ: 9587}

\begin{document}
\maketitle

Το   modem   (Modulator-Demodulator)   είναι   μια   συσκευή   που επιτρέπει
σε   υπολογιστές   να επικοινωνούν μεταξύ  τους  μέσω τηλεφωνικών  γραμμών,
δίνοντας  έτσι  την  ευκαιρία  στους  χρήστες  να έχουν άμεση και εύκολη
πρόσβαση σε πολλές υπηρεσίες. Το modem παίρνει την πληροφορία από τον υπολογιστή
και την μετατρέπει σε ένα σήμα που μπορεί να μεταφερθεί μέσω τηλεφωνικών
γραμμών. Η πληροφορία στο εσωτερικό του υπολογιστή είναι αποθηκευμένη σε ψηφιακή
μορφή ενώ κατά μήκος των τηλεφωνικών γραμμών μεταδίδεται με τη μορφή αναλογικών
σημάτων.


Υπάρχουν  τρία  είδη  modem:  Το  εσωτερικό  ενσωματώνεται  στο εσωτερικό  του
υπολογιστή  σε  ειδική  υποδοχή  και  είναι  το  πιο φθηνό.  Το εξωτερικό
συνδέεται με καλώδιο στο port που υπάρχει για modem και μερικά από αυτά παρέχουν
και δυνατότητα χρήσης Fax. Τέλος υπάρχει και το ISDN modem το οποίο χρησιμοποιεί
μια ειδική   τηλεφωνική γραμμή,   η   οποία   καλείται   γραμμή   ISDN,
πετυχαίνοντας πολύ υψηλούς ρυθμούς μετάδοσης.


Ένας τρόπος μετάδοσης ψηφιακών δεδομένων μέσω αναλογικών τηλεφωνικών γραμμών
είναι η κωδίκευση μετάθεσης συχνότητας (Frequency Shift Keying). Στην κωδίκευση
μετάθεσης συχνότητας, διαφορετικός τόνος αποδίδει τα διαφορετικά bits. Όταν ένα
Modem τερματικού καλεί το Modem ενός υπολογιστή, μεταδίδει έναν τόνο 1,070 hertz
για κάθε 0 και έναν 1,270 hertz για κάθε 1. Το Modem του κεντρικού υπολογιστή
αντίστοιχα χρησιμοποιεί έναν τόνο  2,025  hertz  για  το  0  και έναν  τόνο
2,225  hertz  για  το  1.  Επειδή  λοιπόν  τα  δύο  Modem, μεταδίδουν
διαφορετικούς τόνους, μπορούν και χρησιμοποιούν τη γραμμή ταυτόχρονα. Ο τρόπος
αυτός  επικοινωνίας,  ονομάζεται  full- duplex, δηλαδή  πλήρως  αμφίδρομος.
Σπανίως,  συναντάμε Modem με δυνατότητα μόνο να λαμβάνουν ή να μεταδίδουν ανά
φορά τα οποία αποκαλούνται half- duplex δηλαδή ημιαμφίδρομα.


Ας υποθέσουμε τώρα ότι δύο Modem 300 bps συνδέονται μεταξύ τους και ότι o
χρήστης του τερματικού πληκτρολογεί το γράμμα "α". Ο χαρακτήρας αυτός, στο
δυαδικό σύστημα αναγνωρίζεται ως 01100001 σύμφωνα  με  τον  κώδικα  ASCII.  Το
τερματικό  μεταβιβάζει  τα  bits του χαρακτήρα που πληκτρολογήθηκε στο Modem
μέσω της σειριακής θύρας.  Το Modem δέχεται τα ψηφιακά δεδομένα και αναλαμβάνει
την αποστολή τους στον κεντρικό υπολογιστή χρησιμοποιώντας την απλή γραμμή του
τηλεφωνικού δικτύου της περιοχής  και μεταδίδοντας την κατάλληλη σειρά τόνων. Το
Modem του υπολογιστή δέχεται τα ηχητικά διαμορφωμένα δεδομένα και τα μεταφράζει
σε ψηφιακά παρέχοντας στο σύστημα τα αποτελέσματα. Ο κεντρικός υπολογιστής
επεξεργάζεται τις ψηφιακές αυτές πληροφορίες και επιστρέφει τα αποτελέσματα μέσω
του δικού του Modem με την αντίστροφη διαδικασία.


Προκειμένου πάντως τα Modem να αποκτήσουν τη δυνατότητα αποστολής και λήψης σε
πολύ μεγαλύτερες ταχύτητες από αυτή των 300 bps, απαιτήθηκαν τεχνικές σαφώς πιο
εξελιγμένες και πιο σύγχρονες  από  την  κωδίκευση  μετάθεσης  συχνότητας.
Βήματα  προς  την  κατεύθυνση  αυτή, αποτέλεσαν η κωδίκευση μετάθεσης φάσης
(Phase Shift Keying) αρχικά και η διαμόρφωση ορθογωνικής τάσης (Quadrature
Amplitude Modulation) στη συνέχεια.


Τα σύγχρονα Modem επίσης έχουν να αντιμετωπίσουν και ένα άλλο πολύ βασικό
ζήτημα. Αυτό είναι η  διόρθωση  λαθών.  Κατά  τη  μεταφορά  των  δεδομένων  μέσω
της  γραμμής  του  τηλεφώνου, διάφορα προβλήματα μπορεί να προκύψουν τα οποία
επιφέρουν την κακή λήψη σήματος από το Modem. Το σήμα αυτό πρέπει φυσικά να
επαναληφθεί εφόσον δεν είναι σωστό. Η αναγνώριση του λάθους γίνεται μέσω μίας
μεθόδου που ονομάζεται parity check, δηλαδή επαλήθευση ισότητας. Το parity check
με λίγα λόγια λειτουργεί επικολλώντας  ένα  bit  στο  τέλος  κάθε  μεταβίβασης.
Ανάλογα με τη λειτουργία του parity check είτε ως odd είτε ως even, το bit στην
κατάληξη κάθε σήματος είναι τέτοιο ώστε να διαμορφώνει ένα άθροισμα από bits
"1", μονό ή ζυγό αντίστοιχα. Το Modem που λαμβάνει το σήμα εξετάζει τον αριθμό
των bits "1", μετά το πέρας κάθε μεταβίβασης και εάν διαπιστώσει ότι δε συμφωνεί
με το προσυμφωνημένο parity,  ζητά  την  επανάληψη  της  αποστολής  του. Με τον
τρόπο αυτό, το μηχάνημα μειώνει σημαντικά τις πιθανότητες κάποιου λάθους στην
αποστολή των δεδομένων και προστατεύει το χρήστη από πιθανά προβλήματα στη
δουλειά του. Στην δικιά μας περίπτωση χρησιμοποιούμε την μέθοδο FCS (frame check
sequence). Το FCS είναι ένα νούμερο που βάζουμε στο τέλος μετά από τα δεδομένα
που στείλαμε και έχει υπολογιστεί από μια συνάρτηση(με input τα δεδομένα) που
είναι γνωστή και στον δέκτη και στο αποστολέα. Έτσι ο δέκτης υπολογίζει και
αυτός το FCS νούμερο και μπορεί να καταλάβει αν έγινε λάθος στην αποστολή.


\end{document}
